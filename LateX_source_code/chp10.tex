\chapter{conclusion}

Although the human race has achieved a lot when it comes to safety measures, there are still a huge number of accidents happening at this exact moment. Driving assistance is an interesting technology that helps to make driving and commuting a much more safe experience and that was the motive for choosing this project. V2V technology in its simplest form allows vehicles to communicate with each other in a safe way, data communication is crucial when it comes to safety, with the right data drivers can avoid deadly accidents.
There are a lot of V2V implementations, to each its own advantages, this implementation's main advantage is its reduced cost.

The system consists of two units:
\begin{itemize}
    \item Vehicle unit: This is the unit that’s concerned with getting data from the vehicle.
    \item Server unit: This is the unit responsible for communication with the outer world.
\end{itemize}


At first An overview of the project structure was given and the units that constitutes project, after that each unit was discussed in detail to figure out how it will be built, the next step was to figure out how these units will communicate with each other to reach harmonious system, once each unit is implemented, the next logical step was to test these units by building a prototype that will show the functionality of the project, after making sure that the project is ready to be used in real world applications, some of these applications were discussed and possible future improvements. 

In the end a simple V2V implementation was built with a server that’s represented by a python program that can run on any computer, a PCB that theoretically can be connected to any vehicle.
